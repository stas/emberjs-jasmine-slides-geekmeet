\documentclass[compress]{beamer}

\beamertemplatetransparentcovered
\setbeamertemplate{navigation symbols}{}
\setbeamertemplate{footline}
{
  \leavevmode%
  \hbox{%
    \begin{beamercolorbox}[wd=.3333\paperwidth,left]{}%
      \insertshortinstitute
    \end{beamercolorbox}%
    \begin{beamercolorbox}[wd=.3333\paperwidth,center]{}%
      \insertframenumber{} / \inserttotalframenumber
    \end{beamercolorbox}
    \begin{beamercolorbox}[wd=.3333\paperwidth,right]{}%
      \insertshortdate{}
      \;
    \end{beamercolorbox}
  }%
}

\usepackage{listings}
\usepackage{color}
\usepackage{minted}
\usepackage{hyperref}
\usepackage{textcomp}
\usepackage{marvosym}
\usepackage{xcolor}
\usepackage{graphicx}

\usetheme[light,accent=violet]{solarized}

\usepackage{fontspec}
\setmainfont[Mapping=tex-text]{Source Sans Pro}
\setmonofont[Mapping=tex-text]{Inconsolata}

\graphicspath{{./images/}}

\title{Ember.js and Jasmine BDD.}
\subtitle{Building TodoMVC project.}

\author{Stas Sușcov (\href{mailto:stas@nerd.ro}{stas@nerd.ro})}
\date{Octomber 20th, 2012}

\institute{GeekMeet \#12, Cluj-Napoca, Transilvania}

\begin{document}

% ----------------------------------------------------------------------------
% *** Titlepage <<<
% ----------------------------------------------------------------------------
\maketitle
% ----------------------------------------------------------------------------
% *** END of Titlepage >>>
% ----------------------------------------------------------------------------

% ----------------------------------------------------------------------------
% *** Frame <<<
% ----------------------------------------------------------------------------
\begin{frame}
\frametitle{About Stas}

\begin{itemize}[<+->]
  \item a \href{http://stas.nerd.ro}{nerd}
  \item picky developer
  \item interests: web/operations
  \item I do not do front-end or design
  \item I rewrote and maintain TodoMVC Ember.js ports
  \item (\Heart (food wine cycling))
\end{itemize}
\end{frame}
% ----------------------------------------------------------------------------
% *** END of Frame >>>
% ----------------------------------------------------------------------------

% ----------------------------------------------------------------------------
% *** Frame <<<
% ----------------------------------------------------------------------------
\begin{frame}
  \begin{center}
    \huge Saying \emph{JavaScript} today, I'm not referring to Node.js, CoffeeScript or LiveScript!
  \end{center}
\end{frame}
% ----------------------------------------------------------------------------
% *** END of Frame >>>
% ----------------------------------------------------------------------------

% ----------------------------------------------------------------------------
% *** Frame <<<
% ----------------------------------------------------------------------------
\begin{frame}
\frametitle{Our story}

\hypertarget{why}{}

\begin{itemize}[<+->]
  \item we (I was hired by \href{https://github.com/chevah}{Chevah}) were looking for a front-end solution for non-front-end developers
  \item it had to be MVC
  \item it had to be something which was around for some time already
  \item it had to know \textbf{fancy} stuff like:
    \begin{itemize}[<+->]
      \item bindings
      \item observers
      \item routing
      \item little or almost no HTML (declarative views)
      \item features over modularity (should not be \emph{spread} over into a bunch of separate projects)
    \end{itemize}
\end{itemize}
\end{frame}
% ----------------------------------------------------------------------------
% *** END of Frame >>>
% ----------------------------------------------------------------------------

% ----------------------------------------------------------------------------
% *** Frame <<<
% ----------------------------------------------------------------------------
\begin{frame}

\frametitle{What's the story with TodoMVC}

\begin{itemize}[<+->]
  \item has the mission to help developers understand and pick the best of available JavaScript frameworks
    \begin{itemize}[<+->]
      \item mostly MV* projects
    \end{itemize}
  \item has a strict list of specs
  \item it will also offer tests (WIP)
  \item has some nice guys as developers/maintainers
    \begin{itemize}[<+->]
      \item \href{https://twitter.com/addyosmani}{Addy Osmani}
      \item \href{https://twitter.com/sindresorhus}{Sindre Sorhus}
    \end{itemize}
  \item collaborates directly with project authors when it comes to reviews/design
  \item awesome educational project
\end{itemize}

\begin{flushright}
  \colorbox{white}{\includegraphics[height=0.5cm]{todomvc.png}}
  \\
  \tiny \url{http://todomvc.com}
\end{flushright}

\end{frame}
% ----------------------------------------------------------------------------
% *** END of Frame >>>
% ----------------------------------------------------------------------------

% ----------------------------------------------------------------------------
% *** Frame <<<
% ----------------------------------------------------------------------------
\begin{frame}

\frametitle{Why we picked Ember.js}

\begin{itemize}[<+->]
  \item because of all you can find \hyperlink{why}{above \MoveUp}
  \item has conventions (\emph{cough-cough rails})
  \item it is everything you need in one box
  \item cause it's fun to write Ember.js app, and I'll try to prove it
\end{itemize}

\begin{flushright}
  \includegraphics[width=2cm]{emberjs.png}
  \\
  \tiny \url{http://emberjs.com}
\end{flushright}

\end{frame}
% ----------------------------------------------------------------------------
% *** END of Frame >>>
% ----------------------------------------------------------------------------

% ----------------------------------------------------------------------------
% *** Frame <<<
% ----------------------------------------------------------------------------
\begin{frame}

\frametitle{Ember.js Blog app (almost)}

\begin{itemize}[<+->]
  \item to be honest, I wrote the TodoMVC port while learning Ember.js
  \item to make it easier for everyone, I wrote a simple app so you can understand it just by reading my slides
  \item your next step will be to check out the code from this repository
  \item after what I would like to invite you to take a look at TodoMVC Ember.js port
  \begin{itemize}[<+->]
    \item you should try it and maybe send some patches, hints:
    \begin{itemize}[<+->]
      \item refactor some tests
      \item update the code with upcoming Ember.js v.1.0 API changes
      \item rewrite \texttt{todo.js} model
    \end{itemize}
  \end{itemize}
\end{itemize}

\end{frame}
% ----------------------------------------------------------------------------
% *** END of Frame >>>
% ----------------------------------------------------------------------------

% ----------------------------------------------------------------------------
% *** Frame <<<
% ----------------------------------------------------------------------------
\begin{frame}

\frametitle{app.js}

\begin{itemize}[<+->]
  \item bootstraps our application
  \item \texttt{ApplicationController} and \texttt{ApplicationView} are mandatory!
  \item \texttt{rootElement} is to point the app where it should append views/content
  \item  \texttt{connectOutlets} uses \texttt{\{\{outlet\}\}}, see below
\end{itemize}

\inputminted[fontsize=\tiny,gobble=2,linenos=true,firstline=6,lastline=16]{javascript}{code/js/app.js}

\end{frame}
% ----------------------------------------------------------------------------
% *** END of Frame >>>
% ----------------------------------------------------------------------------

% ----------------------------------------------------------------------------
% *** Frame <<<
% ----------------------------------------------------------------------------
\begin{frame}

\frametitle{app/router.js}

\begin{itemize}[<+->]
  \item \texttt{root} route is mandatory and it is where the app will route first
  \item \texttt{enter} gets called when the route is entered
  \item \texttt{connectOutlets} gets called when route controllers/views have to be connected
\end{itemize}


\inputminted[fontsize=\tiny,gobble=2,linenos=true,firstline=6,lastline=29]{javascript}{code/js/app/router.js}

\end{frame}
% ----------------------------------------------------------------------------
% *** END of Frame >>>
% ----------------------------------------------------------------------------

% ----------------------------------------------------------------------------
% *** Frame <<<
% ----------------------------------------------------------------------------
\begin{frame}

\frametitle{app/model/article.js}

\begin{itemize}[<+->]
  \item is our model (\texttt{ActiveRecord} class if you want)
  \item \texttt{.property()} calls define attributes as a computed property
  \begin{itemize}[<+->]
    \item it is available for other Ember.js components as well
  \end{itemize}
  \item \texttt{init} is the constructor (like \texttt{new, initialize, \_\_construct()})
  \item some class methods like:
  \begin{itemize}[<+->]
    \item \texttt{find} is part of the conventions and used by router to query a specific model object
    \item \texttt{all} is also part of the conventions and used by router to query all available model objects
  \end{itemize}
\end{itemize}

\begin{columns}
\begin{column}{0.5\textwidth}
\inputminted[fontsize=\tiny,gobble=2,linenos=true,firstline=6,lastline=20]{javascript}{code/js/app/model/article.js}
\end{column}
\begin{column}{0.5\textwidth}
\inputminted[fontsize=\tiny,gobble=2,linenos=true,firstline=30,lastline=41,firstnumber=24]{javascript}{code/js/app/model/article.js}
\end{column}
\end{columns}

\end{frame}
% ----------------------------------------------------------------------------
% *** END of Frame >>>
% ----------------------------------------------------------------------------

% ----------------------------------------------------------------------------
% *** Frame <<<
% ----------------------------------------------------------------------------
\begin{frame}

\frametitle{app/controller/publish.js}

\begin{itemize}[<+->]
  \item known principle: skinny controllers, fat models
  \item controller knows about its router and view, use it to handle app logic
\end{itemize}

\inputminted[fontsize=\tiny,gobble=2,linenos=true,firstline=6,lastline=19]{javascript}{code/js/app/controller/publish.js}

\end{frame}
% ----------------------------------------------------------------------------
% *** END of Frame >>>
% ----------------------------------------------------------------------------

% ----------------------------------------------------------------------------
% *** Frame <<<
% ----------------------------------------------------------------------------
\begin{frame}

\frametitle{app/view/publish.js}

\begin{itemize}[<+->]
  \item views can either represent one element view or can contain a set of other views
  \begin{itemize}[<+->]
    \item see \texttt{childViews}, \texttt{Ember.TextField}, \texttt{Ember.TextArea}, \texttt{Ember.ContainerView}, etc.
  \end{itemize}
  \item any attribute ending with \texttt{Binding}, upon instantiation will become a binding to that path
  \begin{itemize}[<+->]
    \item it is available for other Ember.js components as well
  \end{itemize}
  \item ideally, views will generate your HTML (my opinion)
  \item views also handle events, see \texttt{click}, \texttt{insertNewline}, etc.
\end{itemize}

\inputminted[fontsize=\tiny,gobble=2,linenos=true,firstline=6,lastline=21]{javascript}{code/js/app/view/publish.js}

\end{frame}
% ----------------------------------------------------------------------------
% *** END of Frame >>>
% ----------------------------------------------------------------------------

% ----------------------------------------------------------------------------
% *** Frame <<<
% ----------------------------------------------------------------------------
\begin{frame}

\frametitle{app/view/article.js}

\begin{itemize}[<+->]
  \item this view has declarative HTML stored in \texttt{template} attribute
  \item you can define Handlebars.js templates either this way or directly in your HTML file
  \item views can also have a \texttt{layout} attribute (see \texttt{yeild} helper )
  \item views can have helpers, see \texttt{action}, \texttt{log}, \texttt{bindAttr}, etc.
\end{itemize}

\inputminted[fontsize=\tiny,gobble=2,linenos=true,firstline=6,lastline=10]{javascript}{code/js/app/view/article.js}

\end{frame}
% ----------------------------------------------------------------------------
% *** END of Frame >>>
% ----------------------------------------------------------------------------

% ----------------------------------------------------------------------------
% *** Frame <<<
% ----------------------------------------------------------------------------
\begin{frame}
  \begin{center}
  \huge Writing tests in JavaScript compared to other programming languages is probably the most easy
    (it might be easier in Ruby), so you have no excuse not to write tests!
  \end{center}
\end{frame}
% ----------------------------------------------------------------------------
% *** END of Frame >>>
% ----------------------------------------------------------------------------

% ----------------------------------------------------------------------------
% *** Frame <<<
% ----------------------------------------------------------------------------
\begin{frame}

\frametitle{Testing tips}

\begin{itemize}[<+->]
  \item I prefer Jasmine because\ldots it's just one library and it's easy to set up.
  \begin{itemize}[<+->]
    \item for \em{cowboys}, check out \href{https://github.com/rwldrn/idiomatic.js/\#test-facility}{other solutions}
  \end{itemize}
  \item always try to use tools to save time, in our case, I suggest you to take a look at Phantom.js
  \begin{itemize}[<+->]
    \item tests can be run pragmatically, this saves a lot of time!
    \item to keep the project sane and healthy, use some continuous integration solution!
    \item take a look at: \texttt{guard-phantomjs-jasmine}
  \end{itemize}
\end{itemize}

\begin{flushright}
  \colorbox{white}{\includegraphics[height=1.1cm]{jasmine.png}}
  \\
  \tiny \url{http://pivotal.github.com/jasmine/}
\end{flushright}


\end{frame}
% ----------------------------------------------------------------------------
% *** END of Frame >>>
% ----------------------------------------------------------------------------

% ----------------------------------------------------------------------------
% *** Frame <<<
% ----------------------------------------------------------------------------
\begin{frame}

\frametitle{Write unit tests for models}

\begin{itemize}[<+->]
  \item use \texttt{describe} to explain what you are testing
  \item use \texttt{it} to describe the test case
  \item use \texttt{beforeEach} to DRY aka produce cleaner tests and be fast
  \item use multiple \texttt{describe} calls if you want to change the context
  \begin{itemize}[<+->]
    \item say, you want to test class methods where before you were testing instance methods
  \end{itemize}
\end{itemize}

\inputminted[fontsize=\tiny,gobble=2,linenos=true,firstline=3,lastline=23]{javascript}{code/js/app/test/model/article.js}

\end{frame}
% ----------------------------------------------------------------------------
% *** END of Frame >>>
% ----------------------------------------------------------------------------

% ----------------------------------------------------------------------------
% *** Frame <<<
% ----------------------------------------------------------------------------
\begin{frame}

\frametitle{Write integration tests for controllers}

\begin{itemize}[<+->]
  \item use \texttt{spy} calls aka \texttt{stubs}, \texttt{mocks} to isolate tests from other components
  \item don't waste time on writing new (Jasmine) matchers, if your test needs it, rethink it!
\end{itemize}

\inputminted[fontsize=\tiny,gobble=2,linenos=true,firstline=6,lastline=29]{javascript}{code/js/app/test/controller/publish.js}

\end{frame}
% ----------------------------------------------------------------------------
% *** END of Frame >>>
% ----------------------------------------------------------------------------

% ----------------------------------------------------------------------------
% *** Frame <<<
% ----------------------------------------------------------------------------
\begin{frame}
  \begin{center}
  \huge Questions please\ldots
  \\
  Thank you for your time.
  \end{center}
\end{frame}
% ----------------------------------------------------------------------------
% *** END of Frame >>>
% ----------------------------------------------------------------------------

% ----------------------------------------------------------------------------
% *** Frame <<<
% ----------------------------------------------------------------------------
\begin{frame}
\frametitle{Some Links To Check}

\begin{itemize}
  \item TodoMVC -- \url{https://github.com/addyosmani/todomvc/}
  \item Ember.js -- \url{http://emberjs.com/}
  \item Jasmine BDD -- \url{http://pivotal.github.com/jasmine/}
  \item Idiomatic.js -- \url{https://github.com/rwldrn/idiomatic.js/}
  \item Phantom.js -- \url{http://phantomjs.org/}
  \item guard-phantomjs-jasmine -- \url{https://github.com/stas/guard-phantomjs-jasmine}
  \item \url{https://github.com/stas/emberjs-jasmine-slides}
\end{itemize}

\end{frame}
% ----------------------------------------------------------------------------
% *** END of Frame >>>
% ----------------------------------------------------------------------------

\end{document}
