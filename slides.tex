\documentclass[compress]{beamer}

\beamertemplatetransparentcovered
\setbeamertemplate{navigation symbols}{}
\setbeamertemplate{footline}
{
  \leavevmode%
  \hbox{%
    \begin{beamercolorbox}[wd=.3333\paperwidth,left]{}%
      \insertshortinstitute
    \end{beamercolorbox}%
    \begin{beamercolorbox}[wd=.3333\paperwidth,center]{}%
      \insertframenumber{} / \inserttotalframenumber
    \end{beamercolorbox}
    \begin{beamercolorbox}[wd=.3333\paperwidth,right]{}%
      \insertshortdate{}
      \;
    \end{beamercolorbox}
  }%
}

\usepackage{listings}
\usepackage{color}
\usepackage{minted}
\usepackage{hyperref}
\usepackage{textcomp}
\usepackage{marvosym}
\usepackage{xcolor}
\usepackage{graphicx}

\usetheme[dark,accent=violet]{solarized}

\usepackage{fontspec}
\setmainfont[Mapping=tex-text]{Source Sans Pro}
\setmonofont[Mapping=tex-text]{Inconsolata}

\graphicspath{{./images/}}

\title{Ember.js și Jasmine BDD.}
\subtitle{În contextul proiectului TodoMVC.}

\author{Stas Sușcov (\href{mailto:stas@nerd.ro}{stas@nerd.ro})}
\date{Octomber 20th, 2012}

\institute{GeekMeet \#12, Cluj-Napoca, Transilvania}

\begin{document}

% ----------------------------------------------------------------------------
% *** Titlepage <<<
% ----------------------------------------------------------------------------
\maketitle
% ----------------------------------------------------------------------------
% *** END of Titlepage >>>
% ----------------------------------------------------------------------------


\section{Despre}

\subsection{Despre autor}

% ----------------------------------------------------------------------------
% *** Frame <<<
% ----------------------------------------------------------------------------
\begin{frame}
\frametitle{Despre mine}

\begin{itemize}[<+->]
  \item un \href{http://stas.nerd.ro}{nerd}
  \item dezvoltator pretențios și pedant
  \item interese: web/operations
  \item nu fac frontend sau design
  \item (\Heart (papa vin bicicleta))
\end{itemize}
\end{frame}
% ----------------------------------------------------------------------------
% *** END of Frame >>>
% ----------------------------------------------------------------------------

% ----------------------------------------------------------------------------
% *** Frame <<<
% ----------------------------------------------------------------------------
\begin{frame}
  \begin{center}
  \huge Astăzi când zic JavaScript, nu mă refer la Node.js, CoffeeScript sau LiveScript!
  \end{center}
\end{frame}
% ----------------------------------------------------------------------------
% *** END of Frame >>>
% ----------------------------------------------------------------------------

\subsection{Despre problemă}

% ----------------------------------------------------------------------------
% *** Frame <<<
% ----------------------------------------------------------------------------
\begin{frame}
\frametitle{Povestea mea}

\hypertarget{why}{}

\begin{itemize}[<+->]
  \item căutam o soluție pentru frontend pentru non-frontend-iști
  \item sa știe MVC
  \item sa fie populară insa nu pre recentă
  \item sa știe chestii \textbf{fancy} gen:
    \begin{itemize}[<+->]
      \item binding-uri
      \item observere
      \item rutare
      \item sa nu necesite HTML/nu fie foarte declarativ
      \item sa nu fie \em împrăștiat
    \end{itemize}
\end{itemize}
\end{frame}
% ----------------------------------------------------------------------------
% *** END of Frame >>>
% ----------------------------------------------------------------------------

\subsection{Despre resurse}

% ----------------------------------------------------------------------------
% *** Frame <<<
% ----------------------------------------------------------------------------
\begin{frame}

\frametitle{Treaba cu TodoMVC}

\begin{itemize}[<+->]
  \item are ca scop să prezinte cât mai multe librării (framework) JavaScript
    \begin{itemize}[<+->]
      \item în mare măsură MV*
    \end{itemize}
  \item urmează strict un set de specificații
  \item se propune să oferă și teste (însă se lucrează aici)
  \item are nișe persoane interesante la colaboratori
    \begin{itemize}[<+->]
      \item \href{https://twitter.com/addyosmani}{Addy Osmani}
      \item \href{https://twitter.com/sindresorhus}{Sindre Sorhus}
    \end{itemize}
  \item colaborează cu autorii/dezvoltatorii de bază din comunitățile populare
  \item este un proiect educațional extraordinar
\end{itemize}

\begin{flushright}
  \colorbox{white}{\includegraphics[height=0.5cm]{todomvc.png}}
  \\
  \tiny \url{http://todomvc.com}
\end{flushright}

\end{frame}
% ----------------------------------------------------------------------------
% *** END of Frame >>>
% ----------------------------------------------------------------------------

\subsection{Ember.js}

% ----------------------------------------------------------------------------
% *** Frame <<<
% ----------------------------------------------------------------------------
\begin{frame}

\frametitle{Am ales Ember.js}

\begin{itemize}[<+->]
  \item pentru că \hyperlink{why}{vezi mai sus \MoveUp}
  \item pentru că are convenții (cough-cough rails)
  \item pentru că e doar o librărie, nu zece!
  \item pentru că e fun și mai jos voi încerca să demonstrez acest lucru
\end{itemize}

\begin{flushright}
  \includegraphics[width=2cm]{emberjs.png}
  \\
  \tiny \url{http://emberjs.com}
\end{flushright}

\end{frame}
% ----------------------------------------------------------------------------
% *** END of Frame >>>
% ----------------------------------------------------------------------------

\section{Cod}
\subsection{Scop}

% ----------------------------------------------------------------------------
% *** Frame <<<
% ----------------------------------------------------------------------------
\begin{frame}

\frametitle{Demo Ember.js (tentativă de blog)}

\begin{itemize}[<+->]
  \item am scris o aplicație simplă pentru a trezi interesul vostru
  \item unde următorul pas ar fi să vă jucați cu codul din prezentare
  \item și să-i dați o șansă proiectului Ember.js din suita TodoMVC
  \begin{itemize}[<+->]
    \item este loc de niște contribuții cum ar fi
    \begin{itemize}[<+->]
      \item mai multe teste
      \item actualizat cu modificările de API
      \item rescris Todo.js (modelul)
    \end{itemize}
  \end{itemize}
\end{itemize}

\end{frame}
% ----------------------------------------------------------------------------
% *** END of Frame >>>
% ----------------------------------------------------------------------------

\subsection{app.js}

% ----------------------------------------------------------------------------
% *** Frame <<<
% ----------------------------------------------------------------------------
\begin{frame}

\frametitle{app.js}

\begin{itemize}[<+->]
  \item are ca scop instanțierea aplicației
  \item \texttt{ApplicationController} și \texttt{ApplicationView} sunt musai!
  \item \texttt{rootElement} e pentru a specifica unde să-și facă de cap aplicația
  \item la \texttt{\{\{outlet\}\}} se leagă \texttt{connectOutlets}
\end{itemize}

\inputminted[fontsize=\tiny,gobble=2,linenos=true,firstline=6,lastline=16]{javascript}{code/js/app.js}

\end{frame}
% ----------------------------------------------------------------------------
% *** END of Frame >>>
% ----------------------------------------------------------------------------

\subsection{app/router.js}

% ----------------------------------------------------------------------------
% *** Frame <<<
% ----------------------------------------------------------------------------
\begin{frame}

\frametitle{app/router.js}

\begin{itemize}[<+->]
  \item ruta \texttt{root} e musai și de acolo va porni aplicația implicit
  \item \texttt{enter} se apelează când ruta este accesată
  \item \texttt{connectOutlets} se apelează când ruta e accesată pentru a conecta controllere specifice
\end{itemize}


\inputminted[fontsize=\tiny,gobble=2,linenos=true,firstline=6,lastline=29]{javascript}{code/js/app/router.js}

\end{frame}
% ----------------------------------------------------------------------------
% *** END of Frame >>>
% ----------------------------------------------------------------------------

\subsection{app/model/article.js}

% ----------------------------------------------------------------------------
% *** Frame <<<
% ----------------------------------------------------------------------------
\begin{frame}

\frametitle{app/model/article.js}

\begin{itemize}[<+->]
  \item este modelul aplicației (clasă \texttt{ActiveRecord} dacă doriți)
  \item \texttt{.property()} sunt definește proprietăți computate (atribute dinamice)
  \begin{itemize}[<+->]
    \item este valabil pentru oricare componentă din Ember.js
  \end{itemize}
  \item \texttt{init} este constructorul (\texttt{new, initialize, \_\_construct()})
  \item metodele clasei
  \begin{itemize}[<+->]
    \item \texttt{find} face parte din convenție și este apelat de rută dacă e cazul
    \item \texttt{all} este la fel parte din convenție, apelat la fel de rută dacă e cazul
  \end{itemize}
\end{itemize}

\begin{columns}
\begin{column}{0.5\textwidth}
\inputminted[fontsize=\tiny,gobble=2,linenos=true,firstline=6,lastline=20]{javascript}{code/js/app/model/article.js}
\end{column}
\begin{column}{0.5\textwidth}
\inputminted[fontsize=\tiny,gobble=2,linenos=true,firstline=30,lastline=41,firstnumber=24]{javascript}{code/js/app/model/article.js}
\end{column}
\end{columns}

\end{frame}
% ----------------------------------------------------------------------------
% *** END of Frame >>>
% ----------------------------------------------------------------------------

\subsection{app/controller/publish.js}

% ----------------------------------------------------------------------------
% *** Frame <<<
% ----------------------------------------------------------------------------
\begin{frame}

\frametitle{app/controller/publish.js}

\begin{itemize}[<+->]
  \item controllerele sunt simple, modelele ar trebui să fie baza
  \item controllerele conțin logica din view-uri, știu de modele și de rute
\end{itemize}

\inputminted[fontsize=\tiny,gobble=2,linenos=true,firstline=6,lastline=19]{javascript}{code/js/app/controller/publish.js}

\end{frame}
% ----------------------------------------------------------------------------
% *** END of Frame >>>
% ----------------------------------------------------------------------------

\subsection{app/view/publish.js}

% ----------------------------------------------------------------------------
% *** Frame <<<
% ----------------------------------------------------------------------------
\begin{frame}

\frametitle{app/view/publish.js}

\begin{itemize}[<+->]
  \item poate conține mai multe șabloane/view-uri care pot fi diferite
  \begin{itemize}[<+->]
    \item vezi \texttt{childViews}, \texttt{Ember.TextField}, \texttt{Ember.TextArea}, \texttt{Ember.ContainerView}, etc.
  \end{itemize}
  \item orice atribut care se termină în \texttt{Binding}, reprezintă o referință la calea specificată
  \begin{itemize}[<+->]
    \item este valabil pentru oricare componentă din Ember.js
  \end{itemize}
  \item ideal, șabloanele generează HTML-ul
  \item șabloanele știu de evenimente, exemplu: \texttt{click}, \texttt{insertNewline}, etc.
\end{itemize}

\inputminted[fontsize=\tiny,gobble=2,linenos=true,firstline=6,lastline=21]{javascript}{code/js/app/view/publish.js}

\end{frame}
% ----------------------------------------------------------------------------
% *** END of Frame >>>
% ----------------------------------------------------------------------------

\subsection{app/view/article.js}

% ----------------------------------------------------------------------------
% *** Frame <<<
% ----------------------------------------------------------------------------
\begin{frame}

\frametitle{app/view/article.js}

\begin{itemize}[<+->]
  \item este un șablon care folosește un cod HTML predefinit (\texttt{template})
  \item orice șablon poate folosi cod cu sintaxa Handlebars.js
  \item poate avea și un \texttt{layout}
  \item poate folosi helpere, exemplu: \texttt{action}, \texttt{log}, \texttt{bindAttr}, etc.
\end{itemize}

\inputminted[fontsize=\tiny,gobble=2,linenos=true,firstline=6,lastline=10]{javascript}{code/js/app/view/article.js}

\end{frame}
% ----------------------------------------------------------------------------
% *** END of Frame >>>
% ----------------------------------------------------------------------------

\section{Despre testare}

% ----------------------------------------------------------------------------
% *** Frame <<<
% ----------------------------------------------------------------------------
\begin{frame}
  \begin{center}
  \huge A scrie teste pe JavaScript e mai simplu decât în orice alt limbaj de programare (discutabil Ruby), deci nu ai scuze pentru a nu scrie teste!
  \end{center}
\end{frame}
% ----------------------------------------------------------------------------
% *** END of Frame >>>
% ----------------------------------------------------------------------------

% ----------------------------------------------------------------------------
% *** Frame <<<
% ----------------------------------------------------------------------------
\begin{frame}

\frametitle{Testarea aplicației}

\begin{itemize}[<+->]
  \item prefer Jasmine BDD pentru că \ldots este doar o librărie, nu zece!
  \begin{itemize}[<+->]
    \item pentru \em{cowboys}, vezi alte \href{https://github.com/rwldrn/idiomatic.js/\#test-facility}{opțiuni}
  \end{itemize}
  \item tot timpul folosiți unelte pentru a câștiga timp, in cazul nostru, Phantom.js
  \begin{itemize}[<+->]
    \item testele de browser pot fi rulate pragmatic pentru a câștiga timp
    \item pentru a întreține proiectul sănătos, folosiți un serviciu de CI (integrare continouă)
    \item exemplu: \texttt{guard-phantomjs-jasmine}
  \end{itemize}
\end{itemize}

\begin{flushright}
  \colorbox{white}{\includegraphics[height=1.1cm]{jasmine.png}}
  \\
  \tiny \url{http://pivotal.github.com/jasmine/}
\end{flushright}


\end{frame}
% ----------------------------------------------------------------------------
% *** END of Frame >>>
% ----------------------------------------------------------------------------

% ----------------------------------------------------------------------------
% *** Frame <<<
% ----------------------------------------------------------------------------
\begin{frame}

\frametitle{Unit teste pentru modele}

\begin{itemize}[<+->]
  \item \texttt{describe} pentru a specifica ce testezi
  \item \texttt{it} pentru a descrie testul
  \item \texttt{beforeEach} pentru a nu te repeta
  \item folosește mai multe \texttt{describe}-uri pentru a specifica schimbările de context
  \begin{itemize}[<+->]
    \item exemplu, testezi metodele clasei și nu a instanței
  \end{itemize}
\end{itemize}

\inputminted[fontsize=\tiny,gobble=2,linenos=true,firstline=3,lastline=23]{javascript}{code/js/app/test/model/article.js}

\end{frame}
% ----------------------------------------------------------------------------
% *** END of Frame >>>
% ----------------------------------------------------------------------------

% ----------------------------------------------------------------------------
% *** Frame <<<
% ----------------------------------------------------------------------------
\begin{frame}

\frametitle{Teste de integrare pentru controllere}

\begin{itemize}[<+->]
  \item folosește \texttt{spy}-uri (șpioni) sau altfel numiți \texttt{stub}, \texttt{mocks} pentru a izola testele
  \item dacă Jasmine nu are suport pentru eșteptările (expectations) unui test, rescrie testul
\end{itemize}

\inputminted[fontsize=\tiny,gobble=2,linenos=true,firstline=6,lastline=29]{javascript}{code/js/app/test/controller/publish.js}

\end{frame}
% ----------------------------------------------------------------------------
% *** END of Frame >>>
% ----------------------------------------------------------------------------

\section{Încheiere}

% ----------------------------------------------------------------------------
% *** Frame <<<
% ----------------------------------------------------------------------------
\begin{frame}
  \begin{center}
  \huge Întrebări?!
  \\
  Mulțumim pentru atenție!
  \end{center}
\end{frame}
% ----------------------------------------------------------------------------
% *** END of Frame >>>
% ----------------------------------------------------------------------------

\subsection{Contacte și Legături}

% ----------------------------------------------------------------------------
% *** Frame <<<
% ----------------------------------------------------------------------------
\begin{frame}
\frametitle{Legături utile}

\begin{itemize}
  \item TodoMVC -- \url{https://github.com/addyosmani/todomvc/}
  \item Ember.js -- \url{http://emberjs.com/}
  \item Jasmine BDD -- \url{http://pivotal.github.com/jasmine/}
  \item Idiomatic.js -- \url{https://github.com/rwldrn/idiomatic.js/}
  \item Phantom.js -- \url{http://phantomjs.org/}
  \item guard-phantomjs-jasmine -- \url{https://github.com/stas/guard-phantomjs-jasmine}
  \item \url{https://github.com/stas/emberjs-jasmine-slides}
\end{itemize}

\end{frame}
% ----------------------------------------------------------------------------
% *** END of Frame >>>
% ----------------------------------------------------------------------------

\end{document}
